\documentclass{article}

\usepackage{graphicx}
\usepackage{listings}
\usepackage{xcolor}
\usepackage{blindtext}
\lstset { %
  language=C++,
  tabsize=2, 
  showspaces=false, 
  showstringspaces=false, 
  backgroundcolor=\color{black!5}, 
  float=[htb], 
  captionpos=b, 
  basicstyle=\tiny, 
  frame=tbrl, %t: top, r, b, l 
  frameround=tttt, 
  numbers=left, 
  numberstyle=\tiny, 
  numberblanklines=false, 
  linewidth=1.0\textwidth,
  xleftmargin=0.15cm
}

\graphicspath{ {../output/png/} }

\title{TP x : }
\date{\today}
\author{VIAL Sébastien}

\begin{document}
\maketitle
\tableofcontents
\newpage

\section{Hello !}
Ce document contient des extraits de code super cool pour par exemple créer des figures !

\begin{figure}[ht]
  \begin{center}
    \includegraphics[width=0.5\textwidth]{baboon.png}
  \end{center}
  \caption{Légende de la figure}
  \label{fig:baboon}
\end{figure}

\begin{lstlisting}
void codeCpp();
\end{lstlisting}

\end{document}
